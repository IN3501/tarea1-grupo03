\documentclass[preprintnumbers,superscriptaddress,amsmath,%
  amssymb,aps,reprint,nofootinbib]{revtex4-1}

\setlength{\topmargin}{-1.7cm}

\usepackage[utf8]{inputenc}
\usepackage{amsmath}
\usepackage{amsfonts}
\usepackage{amssymb}
\usepackage{amsthm}
\usepackage{caption}
\usepackage[spanish,es-tabla]{babel}
\usepackage{graphicx}
\usepackage{dcolumn}
\usepackage{hyperref}
\usepackage{verbatim}         %Comentarios

\renewcommand{\labelenumi}{\alph{enumi})}

%                                                                              %
%%%%%%%%%%%%%%%%%%%%%%%%%%%%%%%%%%%%%%%%%%%%%%%%%%%%%%%%%%%%%%%%%%%%%%%%%%%%%%%%
%                         Separación de las multicolumnas                      %
%                                                                              %
\setlength{\columnsep}{1cm}
%                                                                              %
%%%%%%%%%%%%%%%%%%%%%%%%%%%%%%%%%%%%%%%%%%%%%%%%%%%%%%%%%%%%%%%%%%%%%%%%%%%%%%%%
%                               Carpeta Imágenes                               %
%                                                                              %
\graphicspath{ {imagenes/} }
%                                                                              %
%%%%%%%%%%%%%%%%%%%%%%%%%%%%%%%%%%%%%%%%%%%%%%%%%%%%%%%%%%%%%%%%%%%%%%%%%%%%%%%%
%                              Headers and footers                             %
%                                                                              %
%\pagestyle{fancy}
%\fancyhf{}
%\fancyhead[LE,RO]{}        %
%\fancyhead[RE,LO]{}        %
%\cfoot{\thepage}
%                                                                              %
%%%%%%%%%%%%%%%%%%%%%%%%%%%%%%%%%%%%%%%%%%%%%%%%%%%%%%%%%%%%%%%%%%%%%%%%%%%%%%%%
%                      Números de sección en las ecuaciones                    %
%                                                                              %
\numberwithin{equation}{section}
%                                                                              %
%%%%%%%%%%%%%%%%%%%%%%%%%%%%%%%%%%%%%%%%%%%%%%%%%%%%%%%%%%%%%%%%%%%%%%%%%%%%%%%%
%                                Nuevos comandos                               %
%\newcommand{\{nuevo comando}}{\{que es el comando}}
\newcommand{\T}{T^\circ}              %Temperatura
%                                                                              %
%%%%%%%%%%%%%%%%%%%%%%%%%%%%%%%%%%%%%%%%%%%%%%%%%%%%%%%%%%%%%%%%%%%%%%%%%%%%%%%%
%                             Numeración de listas                             %
%                                                                              %
%Cambio del nivel 1 en listas Enumeradas
\renewcommand{\labelenumi}{\alph{enumi})}

%Cambio del nivel 2 en listas Enumeradas
\renewcommand{\labelenumii}{\arabic{enumii})}

%Cambio del nivel 1 en listas Itemizadas
\renewcommand{\labelitemi}{-}

%Cambio del nivel 2 en listas Itemizadas
\renewcommand{\labelitemii}{\scriptsize{$\bullet$}}

%                                                                              %
%%%%%%%%%%%%%%%%%%%%%%%%%%%%%%%%%%%%%%%%%%%%%%%%%%%%%%%%%%%%%%%%%%%%%%%%%%%%%%%%
%                             Figuras y Tablas                                 %

\begin{comment}

Para agregar Figuras o tablas, utilice estos códigos

\begin{table}[h!]
\noindent \begin{centering}
\begin{tabular}{|c|c|c|}
\hline 
 &  &  \\
\hline 
 &  &  \\
\hline 
\end{tabular}
\par\end{centering}
\caption{}
\label{cuadro}
\end{table}


\begin{figure}[h!]
\centering
\includegraphics[width=\columnwidth]{}
\caption{}
\label{}
\end{figure}


\end{comment}


%                                                                              %
%%%%%%%%%%%%%%%%%%%%%%%%%%%%%%%%%%%%%%%%%%%%%%%%%%%%%%%%%%%%%%%%%%%%%%%%%%%%%%%%
%                                  Portada                                     %
%                                                                              %

\begin{document}
\preprint{IN3501-1 Tecnologías de la información para la estadística y la gestión}
\title{\Large{Tarea 1}}
\author{Profesor: Ángel Jimenez \\ Profesor: Iván Díaz  \\ Auxiliar: José Canto \\ Auxiliar: Jorge Pinto \\Auxiliar: Macarena Osorio \\Auxiliar: Javiera Ovalle \\ Ayudante: Carlos Vega \\ Integrante: Carla González \\Integrante: Enrique Cumming \\ Integrante: Diego Fuentes\\ Integrante: David de la Puente \\ Fecha de entrega : 25/08/2019} 
% \email{carla.gonzavi@gmail.com}

\affiliation{Universidad de Chile, Facultad de Ciencias Físicas y Matemáticas, Departamento de Ingeniería Civil Industrial}

%\date{\today}\label{Fecha}

%                                                                              %
%%%%%%%%%%%%%%%%%%%%%%%%%%%%%%%%%%%%%%%%%%%%%%%%%%%%%%%%%%%%%%%%%%%%%%%%%%%%%%%%
%                                                                              %








\maketitle
\onecolumngrid

%                                                                              %
%%%%%%%%%%%%%%%%%%%%%%%%%%%%%%%%%%%%%%%%%%%%%%%%%%%%%%%%%%%%%%%%%%%%%%%%%%%%%%%%
%                                                                              %
\section{Introducción}
%                                                                              %
%%%%%%%%%%%%%%%%%%%%%%%%%%%%%%%%%%%%%%%%%%%%%%%%%%%%%%%%%%%%%%%%%%%%%%%%%%%%%%%%
%                                                                              %

% TERCERA PERSONA Y VOZ DIRECTA

%¿Qué es lo que se sabe acerca del tema antes de realizar el estudio?(contexto y marco teórico)
% Contexto


%¿Por qué su estudio es importante?, ¿Cómo va ayudar su estudio a mejorar lo que ya se conoce?(motivación o propósito)


%¿Qué es lo que se esta estudiando? (objetivos)
 

%¿Qué es lo que se hará para lograr el objetivo?(Plan de trabajo)


% Introducción al Montaje
\noindent 
% ¿Por qué lo hizo?
Con el objetivo de desarrollar la Tarea 1 del curso relativo al proyecto 
% Resumen de las secciones
\medskip
En la sección \textbf{II} se muestran los resultados y se discuten los resultados obtenidos.


%                                                                              %
%%%%%%%%%%%%%%%%%%%%%%%%%%%%%%%%%%%%%%%%%%%%%%%%%%%%%%%%%%%%%%%%%%%%%%%%%%%%%%%%
%                                                                              %


\section{Preguntas y Resultados}

\noindent{\large \bf Problema 1}\\
 
 













\end{document}