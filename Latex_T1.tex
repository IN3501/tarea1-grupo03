\documentclass[preprintnumbers,superscriptaddress,amsmath,%
  amssymb,aps,reprint,nofootinbib]{revtex4-1}

\setlength{\topmargin}{-1.7cm}

\usepackage[utf8]{inputenc}
\usepackage{amsmath}
\usepackage{amsfonts}
\usepackage{amssymb}
\usepackage{amsthm}
\usepackage{caption}
\usepackage[spanish,es-tabla]{babel}
\usepackage{graphicx}
\usepackage{dcolumn}
\usepackage{hyperref}
\usepackage{verbatim}         %Comentarios

\renewcommand{\labelenumi}{\alph{enumi})}

%                                                                              %
%%%%%%%%%%%%%%%%%%%%%%%%%%%%%%%%%%%%%%%%%%%%%%%%%%%%%%%%%%%%%%%%%%%%%%%%%%%%%%%%
%                         Separación de las multicolumnas                      %
%                                                                              %
\setlength{\columnsep}{1cm}
%                                                                              %
%%%%%%%%%%%%%%%%%%%%%%%%%%%%%%%%%%%%%%%%%%%%%%%%%%%%%%%%%%%%%%%%%%%%%%%%%%%%%%%%
%                               Carpeta Imágenes                               %
%                                                                              %
\graphicspath{ {imagenes/} }
%                                                                              %
%%%%%%%%%%%%%%%%%%%%%%%%%%%%%%%%%%%%%%%%%%%%%%%%%%%%%%%%%%%%%%%%%%%%%%%%%%%%%%%%
%                              Headers and footers                             %
%                                                                              %
%\pagestyle{fancy}
%\fancyhf{}
%\fancyhead[LE,RO]{}        %
%\fancyhead[RE,LO]{}        %
%\cfoot{\thepage}
%                                                                              %
%%%%%%%%%%%%%%%%%%%%%%%%%%%%%%%%%%%%%%%%%%%%%%%%%%%%%%%%%%%%%%%%%%%%%%%%%%%%%%%%
%                      Números de sección en las ecuaciones                    %
%                                                                              %
\numberwithin{equation}{section}
%                                                                              %
%%%%%%%%%%%%%%%%%%%%%%%%%%%%%%%%%%%%%%%%%%%%%%%%%%%%%%%%%%%%%%%%%%%%%%%%%%%%%%%%
%                                Nuevos comandos                               %
%\newcommand{\{nuevo comando}}{\{que es el comando}}
\newcommand{\T}{T^\circ}              %Temperatura
%                                                                              %
%%%%%%%%%%%%%%%%%%%%%%%%%%%%%%%%%%%%%%%%%%%%%%%%%%%%%%%%%%%%%%%%%%%%%%%%%%%%%%%%
%                             Numeración de listas                             %
%                                                                              %
%Cambio del nivel 1 en listas Enumeradas
\renewcommand{\labelenumi}{\alph{enumi})}

%Cambio del nivel 2 en listas Enumeradas
\renewcommand{\labelenumii}{\arabic{enumii})}

%Cambio del nivel 1 en listas Itemizadas
\renewcommand{\labelitemi}{-}

%Cambio del nivel 2 en listas Itemizadas
\renewcommand{\labelitemii}{\scriptsize{$\bullet$}}

%                                                                              %
%%%%%%%%%%%%%%%%%%%%%%%%%%%%%%%%%%%%%%%%%%%%%%%%%%%%%%%%%%%%%%%%%%%%%%%%%%%%%%%%
%                             Figuras y Tablas                                 %

\begin{comment}

Para agregar Figuras o tablas, utilice estos códigos

\begin{table}[h!]
\noindent \begin{centering}
\begin{tabular}{|c|c|c|}
\hline 
 &  &  \\
\hline 
 &  &  \\
\hline 
\end{tabular}
\par\end{centering}
\caption{}
\label{cuadro}
\end{table}


\begin{figure}[h!]
\centering
\includegraphics[width=\columnwidth]{}
\caption{}
\label{}
\end{figure}


\end{comment}


%                                                                              %
%%%%%%%%%%%%%%%%%%%%%%%%%%%%%%%%%%%%%%%%%%%%%%%%%%%%%%%%%%%%%%%%%%%%%%%%%%%%%%%%
%                                  Portada                                     %
%                                                                              %

\begin{document}
\preprint{IN3501-1 Tecnologías de la información para la estadística y la gestión}
\title{\Large{Tarea 1}}
\author{Profesor: Ángel Jimenez \\ Profesor: Iván Díaz  \\ Auxiliar: José Canto \\ Auxiliar: Jorge Pinto \\Auxiliar: Macarena Osorio \\Auxiliar: Javiera Ovalle \\ Ayudante: Carlos Vega \\ Integrante: Carla González \\Integrante: Enrique Cumming \\ Integrante: Diego Fuentes\\ Integrante: David de la Puente \\ Fecha de entrega : 25/08/2019} 
% \email{carla.gonzavi@gmail.com}

\affiliation{Universidad de Chile, Facultad de Ciencias Físicas y Matemáticas, Departamento de Ingeniería Civil Industrial}

%\date{\today}\label{Fecha}

%                                                                              %
%%%%%%%%%%%%%%%%%%%%%%%%%%%%%%%%%%%%%%%%%%%%%%%%%%%%%%%%%%%%%%%%%%%%%%%%%%%%%%%%
%                                                                              %








\maketitle
\onecolumngrid

%                                                                              %
%%%%%%%%%%%%%%%%%%%%%%%%%%%%%%%%%%%%%%%%%%%%%%%%%%%%%%%%%%%%%%%%%%%%%%%%%%%%%%%%
%                                                                              %
\section{Introducción}
%                                                                              %
%%%%%%%%%%%%%%%%%%%%%%%%%%%%%%%%%%%%%%%%%%%%%%%%%%%%%%%%%%%%%%%%%%%%%%%%%%%%%%%%
%                                                                              %

% TERCERA PERSONA Y VOZ DIRECTA

%¿Qué es lo que se sabe acerca del tema antes de realizar el estudio?(contexto y marco teórico)
% Contexto


%¿Por qué su estudio es importante?, ¿Cómo va ayudar su estudio a mejorar lo que ya se conoce?(motivación o propósito)


%¿Qué es lo que se esta estudiando? (objetivos)
 

%¿Qué es lo que se hará para lograr el objetivo?(Plan de trabajo)


% Introducción al Montaje
\noindent 
% ¿Por qué lo hizo?
Con el objetivo de desarrollar la Tarea 1 del curso TICS relativo al proyecto semestral de tecnologías de la información, se pretende mostrar en este informe los avances de la etapa de concepción de la idea, identificación del problema y como pretendemos abordarlo durante el semestre. El centro de este proyecto consiste en identificar una problemática u oportunidad de mejora dentro de una empresa, organización o emprendimiento real para aplicar las TICS y  que funcionen como facilitador o potenciador de estas organizaciones.\\
La PYME que el equipo ayudará es "BLAHBLAK", esta organización es de la hermana de una integrante del equipo que ofrece distintos productos de pastelería y banquetería para diferentes tipos de clientes, por un lado ella ofrece pasteles y tortas varias, por otro lado pasteleria sin azucar (para personas con diabetes) y por otro lado servicio de banquetería para eventos tales como matrimonios, bautizos, etc. Entonces el plan consiste en armar una pagina web donde ella pueda por un lado ofrecer sus productos de pastelería y su servicio de banquetería, además de ayudarla a controlar sus finanzas.
% Resumen de las secciones
\medskip
En la sección \textbf{II} se muestran los resultados y se discuten los resultados obtenidos.


%                                                                              %
%%%%%%%%%%%%%%%%%%%%%%%%%%%%%%%%%%%%%%%%%%%%%%%%%%%%%%%%%%%%%%%%%%%%%%%%%%%%%%%%
%                                                                              %


\section{Desarrollo}

\noindent{\large \bf Planteamiento del problema}\\
 
 Fernanda González estudiante de tercer año de Gastronomía Internacional se dedica hace un tiempo a ofrecer distintos tipos de productos/servicios gastronómicos para distintos tipos de clientes, para poder realizar la investigación de sus necesidades de forma correcta se concretó una reunión con ella para que nos contara como se desarrolla su negocio y en que aspectos nosotras/os la podríamos ayudar, a continuación se muestra la segmentación de mercado que Fernanda identifica:
 
\begin{table}[h!]
\noindent \begin{centering}
\begin{tabular}{|c|c|}
\hline 
Cliente & Producto/Servicio & Problemáticas\\
\hline 
Personas generales & Tortas, Cupcakes, Tartas, Pasteles \\ 
\hline
Personas con diabétes &  Tortas, Cupcakes, Tartas, Pasteles [todo sin azucar]\\
\hline
Personales generales & Coctales para eventos \\
\hline
\end{tabular}
\par\end{centering}
\caption{Segmentación de mercado}
\label{cuadro}
\end{table}

Cada segmento de clientes le presenta distintos desafíos a Fernanda, el desafío común para los tres segmentos de cliente es el medio de comunicación producto que todo lo ofrecido es personalizado, es decir, cada persona arma su pastel o banquete como le parezca mejor.\\
El primer segmento el cliente es bastante amplio, cualquier persona que quiera encargar cierto producto de pastelería para alguna ocación en particular, el principal desafío que se enfrenta Fernanda es que al ser un pedido personalizado, las personas pueden pedir distintas combinaciones de productos, por ejemplo; si un cliente quiere una torta de cumpleaños se debe saber para cuantas personas es, si es de biscocho, panqueque, merengue, etc, el tipo relleno y si es que tiene fondan o no para la cubierta de la torta. Entonces para cada tipo de producto ofrecido varían distintos parámetros que debe definir el cliente por lo que cada vez que a ella le llega un pedido debe ocupar mucho de su tiempo hablando con el cliente y muchas veces estos no le responden las preguntas que deben responder o no entienden bien la información.\\

El segundo segmento de cliente corresponde a personas con diabetes o interesados en poder disfrutar de cosas dulces pero sin azucar, le problema principal es muy parecido al del segmento uno pero las restricciones tienen que ver con que la variedad no es tan amplia, ya que no todos los pasteles se pueden cocinar sin azucar, por lo que la dificultad que se presenta es poder calcular la cantidad de carbohidratos por porción, es necesario dar esta información ya que a lasos diabeticas/os este es el dato que les interesa conocer, ahora para poder calcular este valor es necesario saber cuantas porciones tiene el pastel y de qué está hecho pero como los productos son personalizados este valor cambia para cada pedido.\\

El tercer segmento de clientes corresponde a personas que tengan eventos y necesiten un servicio de banquetería (matrominios, bautizos, cumpleaños, etc), el principal complejidad de este segmento es que es muy variado lo que puede necesitar un cliente, por un lado tienes al cliente que quiere que lleves todo al evento; desde la comida hasta los insumos como platos, mesas, cubiertos, meseras/meseros, bar tender, etc. Por otro lado esta el cliente que quiere cierta cantidad de porciones de cierto tipo de coctel y nada más que eso.\\

Por todo lo anterior observamos varias brechas donde podemos actuar, en primer lugar la idea sería crear una pagina donde las personas puedan armar sus pedidos con todas las especificaciones que estos deseen y que la pagina vaya mostrando como varían los precios según las especificaciones que las personas entreguen, en segundo lugar también tener un catalogo de productos estandar para llegar y comprar sin tener que decidir nada (para el segmento de clientes que no quiera decidir y busque algo ya pensado), en tercer para el segmento de clientes de productos sin azucar la pagina debe poder hacer el producto personalizado (parecido al primer segmento) agregándole la variante de la cantidad de carbohidratos por porción, para poder hacer esto es necesario que se ingresen la cantidad de materiales que se utilizarían para cocinar un producto y tener una base de datos con la cantidad de carbohidratos por porción de cada tipo de alimento utilizado para así poder calcular la cantidad de carbohidratos por porción, la particularidad es que esta cantidad va a variar segun los tipos de productos que se utilicen en la preparación y la idea es que se vaya mostrando en pantalla como van variando los carbohidratos por porción, en cuarto lugar una plataforma aparte que permita hacer el presupuesto de un evento completo con todas las variantes que se pueden ofrecer (con o sin insumos) y en último lugar una plataforma que le permita a Fernanda poder controlar sus finanzas, la idea de esta ultima utilidad tiene que ver con que ella pueda ingresar al sistema la cantidad de materiales que utilizó para un producto, a que precio los compró y cuanto tiempo de trabajo le costó hacer el producto, para luego poder hacer estadísticas con todos los datos almacenados, la idea de esta parte de la pagina es que no sea abierta al público y le sirva a Fernadad para poder ordenar sus utilidades, que ella pueda ver que productos le traen mas utilidades, cuales son los productos más vendidos, en que fechas y que tipo de eventos son lo que más le conviene.\\


COMPAÑEROS! AQUI LAS TAREAS: Lean lo anterior pa que entiendan la idea general, si se dan cuenta hay como 4 partes de la pagina y se debe hacer "Buscar en el mercado 2 propuestas similares a las suyas y se˜nalar al menos 3 aspectos positivos
y 3 negativos de cada una" Entonces cada uno tendrá que hacer esta pega para uno de los siguientes puntos

1) segmento personas naturales que quieran comprar pasteles personalizados y estandar (se refiere a estandar ya que habra una seccion donde tu armes tu pastel o otra donde hayan pasteles ya publicados llegar y llevar, notar que es diferente a una pasteleria ya que todo esto es a pedido) + servicios de banqueterías (notar que podrían ser que solo pidan comida o que pidan el servicio completo con mesas, sillas, meseros, etc)


2) pasteleria para diabeticos, centrarse en el tema de mostrar los carbohidratos por porcion, averiguar del uso de la tagatosa (endulzante sin carbohidratos) que es el producto que usa mi hermana para endulsar, averiguar tambien de como son los precios comparados con los precios de un pastel normal, ademas de comparar el precio de 1k de azucar con 1k de tagatosa

R(Diego): Luego de investigar diversas propuestas similares de paginas de pastelerías para diabeticos, las dos más interesantes resultaron ser Pastelería Mango y De Tartas & Tortas. \\
La pagina de Pastelería Mango presenta varios aspectos positivos y negativos, como puntos altos están el buen apoyo visual que brindan las buenas imágenes de pasteles y tortas, otro punto a favor es que cada producto expuesto en la pagina tiene su respectivo precio, también es destacable que los medios contactos son bastante visibles y por ultimo aspecto positivo está que la pagina cuenta con un sistema de reserva de pedidos de pasteles y tortas online, pasando a los aspectos negativos el primero que podemos notar es que el sistema de reservas no permite personalizar productos, otro aspecto negativo notable se ve en que a pesar de exponer las direcciones de las sucursales no hay mapas que muestren la ubicación referencial de estas y por ultimo podemos observar que los productos no cuentan su respectiva información nutricional.\\
Por otra parte esta la pagina de De Tartas & Tortas que al igual que la pagina anteriormente expuesta tiene aspectos negativos y positivos, en aspectos positivos podemos encontrar una amplia gama de productos además de una buena organización de estos(cada cual en su respectiva sección), otro aspecto positivo es que se puede ver es la exposición de los precios de cada producto, además de una visión de cual es la empresa a la que estamos comprando y como aspecto positivo final encontramos que la pagina ofrece contacto especial para empresas que requieran los productos de manera mas masiva, como aspectos negativos volvemos a ver que hay nula información sobre el productos mas allá del nombre de este, otro aspecto negativo es que hay secciones de la pagina que presentan información obsoleta(ofertas y noticias) y como aspecto negativo final notamos que no hay un sistema se reserva de pedidos online.\\
\\
Al comparar las dos paginas anteriores notamos que hay aspectos positivos que hay que tomar en cuenta al momento de realizar nuestro proyecto, así como también hay un aspecto negativo que se repite en ambos casos que es la nula información nutricional sobre los productos que están siendo ofrecidos a los clientes, lo cual no podemos dejar pasar como aspecto a tomar en cuenta por ello tenemos planeado crear un indicador en nuestra pagina que muestre en cada productos cual es la cantidad de carbohidratos por cada porción consumida además de los ingredientes que son utilizados para endulzar las preparaciones, en el caso de "BLAHBLAK" las preparaciones son endulzadas con tagatosa, un monosacárido derivado de la leche, libre de carbohidratos pero que endulza el doble que el azúcar, es ideal para el consumo de gente con diabetes ya que no eleva la glicemia, la tagatosa es ideal para preparaciones pasteleras, pero su único punto en contra de este endulzante es su valor que ronda alrededor de los 6000 pesos el medio kilogramo a diferencia de el azúcar convencional que tiene un costo aproximadamente 20 veces menor, esta diferencia de costos trae como consecuencia inevitable un costo mayor de producción en la pastelería y en los precios de esta en comparación con productos para gente que no posee diabetes.

enlaces; https://www.pasteleriamango.cl  ; https://www.detartasytortas.cl



3) servicio de finanzas para mi hermana, donde ella debe ingresar para cada producto que venda (sea de pasteleria, cocteleria, eventos, sin azucar) cuando gasto, en que cosas y cuanto tiempo le llevo hacerlo, averiguar de que tipo de datos estadisticos le podemos entregar que le vayan  a servir, averiguar sobre como llegar un registro de las fechas tambien para dar estadisticas mes a mes y ver que cosas que venden en que epocas del año, etc. 




\end{document}
